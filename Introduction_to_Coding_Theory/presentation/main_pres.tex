\documentclass[aspectratio=169,10pt]{beamer}
\usepackage[T1]{fontenc}
\usepackage[english]{babel}
\usepackage{graphicx}
\usepackage[utf8]{inputenc}
\usepackage{amsmath}
\usepackage{amsthm}
\usepackage{amsfonts}
\usepackage{amssymb}
\usepackage{mathabx}
\usepackage{mathpazo}
\usepackage{eulervm}
\usepackage{natbib}
\usepackage{tikz}
\usepackage{blkarray}
\usepackage{setspace}
\usepackage{epsfig}
\usepackage{caption}
\usepackage{enumitem}
\usepackage{caption}
\usepackage{subcaption}
\usepackage{float}
\usepackage{bbm}
\usepackage{adjustbox}
\usepackage{url}

\usepackage[citations,definitionLists,hashEnumerators,smartEllipses,tightLists=false,pipeTables,tableCaptions,hybrid]{markdown}
\usepackage{longtable}
%\usepackage[table]{xcolor}
\newcommand{\F}{\mathbb{F}}
\newcommand{\T}{\mathbb{T}}
\newcommand{\K}{\mathbb{K}}
\newcommand{\R}{\mathbb{R}}
\newcommand{\N}{\mathbb{N}}
\newcommand{\Z}{\mathbb{Z}}
\newcommand{\C}{\mathbb{C}}
\newcommand{\Q}{\mathbb{Q}}
\newcommand{\A}{\mathbb{A}}
\def\P{{\mathbb P}}
\def\B{{\mathbf b}}
\def\x{{\mathbf x}}
\def\X{{\mathbb X}}
\def\d{{\delta}}
\def\w{{\mathbf w}}
\def\W{{\mathbf W}}
\def\a{{\mathbf a}}
\def\b{{\mathbf b}}
\newcommand{\sA}{{\mathcal A}}
\newcommand{\sB}{{\mathcal B}}
\newcommand{\sC}{{\mathcal C}}
\newcommand{\sD}{{\mathcal D}}
\newcommand{\sE}{{\mathcal E}}
\newcommand{\sF}{{\mathcal F}}
\newcommand{\sG}{{\mathcal G}}
\newcommand{\sH}{{\mathcal H}}
\newcommand{\sI}{{\mathcal I}}
\newcommand{\sJ}{{\mathcal J}}
\newcommand{\sK}{{\mathcal K}}
\newcommand{\sL}{{\mathcal L}}
\newcommand{\sM}{{\mathcal M}}
\newcommand{\sN}{{\mathcal N}}
\newcommand{\sO}{{\mathcal O}}
\newcommand{\sP}{{\mathcal P}}
\newcommand{\sQ}{{\mathcal Q}}
\newcommand{\sR}{{\mathcal R}}
\newcommand{\sS}{{\mathcal S}}
\newcommand{\sT}{{\mathcal T}}
\newcommand{\sU}{{\mathcal U}}
\newcommand{\sV}{{\mathcal V}}
\newcommand{\sW}{{\mathcal W}}
\newcommand{\sX}{{\mathcal X}}
\newcommand{\sY}{{\mathcal Y}}
\newcommand{\sZ}{{\mathcal Z}}
\newcommand{\nn}{{\nonumber}}
\newcommand{\tres}{\operatorname{res}^\T}
\newcommand{\an}{\text{\rm an}}
\newcommand{\sgn}{\operatorname{sgn}}
\newcommand{\lin}{\operatorname{lin}}
\newcommand{\inter}{\operatorname{int}}
\newcommand{\Vertex}{\operatorname{Vert}}
\newcommand{\Mat}{\operatorname{Mat}}
\newcommand{\supp}{\operatorname{supp}}
\newcommand{\Char}{\operatorname{char}}
\newcommand{\cdeg}{\operatorname{cdeg}}
\newcommand{\spn}{\operatorname{span}}
\newcommand{\St}{\operatorname{St}}
\newcommand{\Tr}{\operatorname{Tr}}
\newcommand{\dv}{\operatorname{div}}
 \newcommand{\Vol}{\operatorname{Vol}}
  \newcommand{\Ehr}{\operatorname{Ehr}}

\def\lm{\textrm{\small LM}}

\newcommand{\M}{\mathbb M}

  
\newcommand{\dis}{\displaystyle}
\newcommand{\Supp}{\operatorname{Supp}}
%\newcommand{\supp}{\operatorname{supp}}
\newcommand{\Spec}{\operatorname{Spec}}
\newcommand{\Hom}{\operatorname{Hom}}
\newcommand{\Ext}{\operatorname{Ext}}
\newcommand{\Ker}{\operatorname{Ker}} 
\newcommand{\im}{\operatorname{Im}} 
\newcommand{\depth}{\operatorname{depth}}
\newcommand{\h}{\operatorname{height}}
%\newcommand{\deg}{\operatorname{deg}}
\newcommand{\codim}{\operatorname{codim}}
\newcommand{\reg}{\operatorname{reg}}
\newcommand{\Cl}{\operatorname{Cl}} 
\newcommand{\Pic}{\operatorname{Pic}}
\newcommand{\rank}{\operatorname{rank}}
\newcommand{\ck}{\check}
\newcommand{\sig}{\sigma}
\newcommand{\Sig}{\Sigma}
\newcommand{\sat}{\operatorname{Sat}}
  \newcommand{\lcm}{\operatorname{lcm}}
\newcommand{\Pp}{\mathbb P}
\newcommand{\Tq}{{ (\F_q^*)}}
\newcommand{\cC}{\cl C}
\DeclareMathOperator{\red}{Red}
\DeclareMathOperator{\den}{den}
\newcommand{\cl}[1]{\mathcal{#1}}
\newcommand{\ov}{\overline}
\newcommand{\la}{\langle}
\newcommand{\ra}{\rangle}
\def\aa{{\bf \alpha}}
\def\bb{\beta}
\def\a{{\bf a}}
\def\t{{\bf t}}
\def\q{{\bf q}}
\def\y{{\bf y}}
\def\z{{\bf z}}
\def\uu{{\bf u}}
\def\vv{{\bf v}}
\def\x{{\bf x}}
\def\m{{\bf m}}
\def\k{{\mathbb k}}
\def\kay{{\chi}}
\def\b{{\bf b}}
\def\e{{\bf e}}
\def\ev{{\text{ev}}}
\newcommand{\cG}{\cl G}
\def\aa{\alpha}
\def\Ta{\Tilde{\alpha}}
\def\Tia{\Tilde{a}}
\def\Tid{\Tilde{d}}

\newcommand{\Or}{\mathcal{O}}
\newcommand{\fp}{{\mathfrak p}}
\newcommand{\fq}{{\mathfrak q}}
\newcommand{\fm}{{\mathfrak m}}

\DeclareMathOperator{\argmin}{argmin}
\DeclareMathOperator{\Span}{Span}
\newcommand{\set}[1]{\left\{#1\right\}}
\newcommand{\size}[1]{\left|#1\right|}
\newcommand{\calX}{\mathcal X}
\newcommand{\calY}{\mathcal Y}

\newcommand{\Floor}[1]{\left\lfloor #1 \right\rfloor}
\newcommand{\Floorfrac}[2]{\left\lfloor\frac{#1}{#2}\right\rfloor}
\newcommand{\Ceil}[1]{\left\lceil #1 \right\rceil}

\newtheorem{remark}[theorem]{Remark}
\newtheorem{proposition}[theorem]{Proposition}
\newtheorem{construction}{Construction}
\newtheorem{conjecture}[theorem]{Conjecture}



%\usetheme{Warsaw}
%\usetheme{default}
%\usetheme{AnnArbor}
%\usetheme{Antibes}
%\usetheme{Bergen}
%\usetheme{Berkeley}
%\usetheme{Berlin}
%\usetheme{Boadilla}
%\usetheme{CambridgeUS}
%\usetheme{Copenhagen}
%\usetheme{Darmstadt}
%\usetheme{Dresden}
%\usetheme{Frankfurt}
%\usetheme{Goettingen}
%\usetheme{Hannover}
%\usetheme{Ilmenau}
%\usetheme{JuanLesPins}
%\usetheme{Luebeck}
%\usetheme{Madrid}
\usefonttheme{serif}
%\usecolortheme{beaver}
%\usecolortheme{crane}--sarı
%\usecolortheme{dolphin}
%\usecolortheme{dove}
%\usecolortheme{fly}
%\usecolortheme{lily}
%\usecolortheme{orchid}
%\usecolortheme{rose}
%\usecolortheme{seagull}
%\usecolortheme{seahorse}
%\usecolortheme{whale}
%\usecolortheme{wolverine}--sarı

\DeclareOptionBeamer{compress}{\beamer@compresstrue}
\ProcessOptionsBeamer

\mode<presentation>

\useoutertheme[footline=authortitle, footline=frame number,subsection=false]{miniframes}
\useinnertheme{circles}
\useoutertheme{tree}
\usecolortheme{dolphin}

\definecolor{beamer@blendedblue}{rgb}{0.137,0.466,0.741}
\definecolor{tealblue}{rgb}{0.21, 0.46, 0.53}
\definecolor{smokyblack}{rgb}{0.06, 0.05, 0.03}
\setbeamercolor{structure}{fg=smokyblack}
\setbeamercolor{titlelike}{parent=structure}
\setbeamercolor{frametitle}{bg=structure!70,fg=white}
\setbeamerfont{frametitle}{size=\small,series=\bfseries}
\setbeamercolor{title}{fg=black}
\setbeamercolor{author}{fg=black}
\setbeamercolor{item}{fg=black}
\setbeamertemplate{page number in head/foot}[totalframenumber] %Pour ajouter le numéro des slides
\setbeamerfont{block title}{size=\small,series=\bfseries}

\setbeamercolor{block title}{fg=white,bg=smokyblack}
\setbeamercolor{block body}{fg=black,bg=smokyblack!10}

\setbeamercolor{block title example}{fg=white,bg=teal}
\setbeamercolor{block body example}{fg=black,bg=teal!10}

\definecolor{glaucous}{rgb}{0.38, 0.51, 0.71}
\definecolor{ochre}{rgb}{0.8, 0.47, 0.13}
\definecolor{chromeyellow}{rgb}{1.0, 0.65, 0.0}
\setbeamercolor{block title alerted}{fg=white,bg=ochre!45!chromeyellow!90}
\setbeamercolor{block body alerted}{fg=black,bg=ochre!95!chromeyellow!20}

\setbeamercolor{frametitle}{bg=tealblue!75,fg=white}

\setbeamercolor{block title theorem}{fg=white,bg=ochre!45!chromeyellow!90}
\setbeamercolor{block body theorem}{fg=black,bg=ochre!95!chromeyellow!20}

\setbeamercolor{block title definition}{fg=white,bg=ochre!45!chromeyellow!90}
\setbeamercolor{block body definition}{fg=black,bg=ochre!95!chromeyellow!20}

\setbeamertemplate{frametitle}{%
	\nointerlineskip%
	\begin{beamercolorbox}[wd=\paperwidth,ht=2.0ex,dp=0.6ex]{frametitle}
		\hspace*{1ex}\insertframetitle%
	\end{beamercolorbox}%
}

\makeatletter
\patchcmd{\slideentry}{\advance\beamer@tempdim by -.05cm}{\advance\beamer@tempdim by\beamer@vboxoffset\advance\beamer@tempdim by\beamer@boxsize\advance\beamer@tempdim by 1.2\pgflinewidth}{}{}
\patchcmd{\slideentry}{\kern\beamer@tempdim}{\advance\beamer@tempdim by 2pt\advance\beamer@tempdim by\wd\beamer@sectionbox\kern\beamer@tempdim}{}{}
\makeatother

\mode<all>
%\includeonlyframes{current}

\setbeamersize{text margin left=0.8cm,text margin right=0.8cm}

\setbeamertemplate{navigation symbols}{}
\makeatletter
\g@addto@macro\normalsize{%
	\setlength\belowdisplayskip{0.2em}
	\setlength\abovedisplayskip{0.2em}
}

%----------------


\colorlet{darkred}{red!80!black}
\colorlet{darkblue}{blue!80!black}
\colorlet{darkgreen}{green!40!black}

\title[Code-Based Cryptography]{Introduction to Code-Based Cryptography}
\author[R. Haykır, F. Gholami Ghahderijani]{Mentees: Ronahi Haykır, Fatemeh Gholami Ghahderijani\\ Boğaziçi  University and Isfahan University}
\institute[]{
Mentor: Dr. Yağmur Çakıroğlu
\\Hacettepe University
}
\date{August 9, 2025}



\begin{document}


\begin{frame}
    \titlepage
\end{frame}


\begin{frame}
    \frametitle{Outline}
    \tableofcontents
\end{frame}


\begin{frame}
    \begin{center}
        \Huge{ \textbf{Part One}}\\
        \vskip1em
        \large{An Introduction on Algebraic Coding Theory As A Basis
For Code-based Cryprography}
    \end{center}
\end{frame}


\section{Algebraic Coding Theory}

\subsection{Hamming-metric Codes}
\begin{frame}[allowframebreaks]
        \begin{block}{Linear Codes}
         Let $1 \leq k \leq n$ be positive integers. $C$ is an $[n, k]$ \textit{linear code} over $\mathbb F_q$, if $C$ is a $k$ dimensional linear subspace of $\mathbb F_q^n$.
        \end{block} 

        \begin{block}{Hamming Weight}
        Let n be a positive integer. The \textit{Hamming weight} of 
        $\textbf{x} \in \mathbb F_q^n$ is given by
        \[
        \mathrm{wt}_H(x) = 
        \left| \{\, i \in \{1, \dots, n\} 
        \mid x_i \neq 0 \,\} \right|.
        \]    
        \end{block}

        \begin{exampleblock}{Example}
            Define 
            \[
                C = 
                \{\,
                (0,0,0,0,0), (1,0,0,0,0), (0,1,0,0,0), (1,1,0,0,0)  
                \,\}
            \]
            $C$ is a $[5, 2]$ linear code over $\mathbb{F}_2$ with 
            $|C| = q^k = 2^2 = 4$.
            \[
                wt_H(c_1)=5,\,wt_H(c_2)=wt_H(c_3)=4,\,wt_H(c_24)=3. 
            \] 
        \end{exampleblock}
        
\framebreak

        \begin{block}{Hamming Distance}
        For $\textbf{x}, \textbf{y} \in \mathbb F_q^n$ the \textit{Hamming distance} between \textbf{x} and \textbf{y} is the number of indices they differ that is given by
        \[
        \mathrm{d}_H(x, y) =
        \left| \{\, i \in \{1, \dots, n\} 
        \mid x_i \neq y_i \,\} \right|.
        \]
        \end{block}

        \begin{alertblock}{Notice}  
            For a linear code, Hamming distance is an induced form of Hamming weight,
            \[
                \mathrm{d}_H(x, y) = \mathrm{wt}_H(x-y).
            \]
        \end{alertblock}

\framebreak

        \begin{block}{Minimum Distance}
        Let $C$ be a code over $\mathbb F_q$. The \textit{minimum Hamming distance} of $C$ is given by
        \[
        \mathrm{d}_H(C) = 
        \min\{\,d_H(\textbf{x}, \textbf{y}) \mid 
        \textbf{x}, \textbf{y} \in C, \textbf{x} \neq \textbf{y} \,\}.
        \]
        \end{block}

        %\begin{block}{Hamming Ball}For a positive integer $r$, we define \emph{Hamming ball} as all vectors in $\mathbb{F}_q^n$ with Hamming weight at most $r$:\[ B_H(r, n, q) =\{\, \mathbf{x} \in \mathbb{F}_q^n \mid  \mathrm{wt}_H(\textbf{x}) \leq \textit{r}\,\}.\] \end{block}

        \begin{exampleblock}{Example}
            $d_H(c_1, c_2)=d_H(c_1, c_3)=1\,d_H(c_1, c_4)=2,\,d_H(c_2, c_3)=2,\,d_H(c_2, c_4)=1,\,d_h(c_3, c_4)=1.$\\
            Thus, $d_H(C)=1$.
        \end{exampleblock}
        
\framebreak

    \begin{block}{Decoding Algorithm}
        The minimum distance of a code is connected to the error correcting capacity of that code. A code can correct up to t errors if for all 
        $\mathbf{x} \in \mathbb{F}_q^n$ with $d_H(\mathbf{x}, C) \leq t$, there is a unique $\mathbf{y} \in C$ such that $d_H(\mathbf{x}, \mathbf{y}) \leq t$.
        We construct an \textit{decoding algorithm} $D$ to find the closest codeword $\mathbf{y} \in C$ such that $d_H(\mathbf{x}, \mathbf{y}) \leq t$ for a given 
        $\mathbf{x} \in \mathbb{F}_q^n$.\\[0.4cm]
    \end{block}
    
\framebreak

    \begin{block}{Theorem 1}
        [Singleton Bound]
        \textit{Let $k \leq n$ be positive integers and $C$ an $[n, k]$ linear code over $\mathbb{F}_q$. Then,}
        \[
        d_H \leq n - k + 1.
        \]        
    \end{block}

    \begin{block}{Proof}
        Let $C$ be a linear code over $\mathbb{F}_q^n$ and define $\varphi: C \xrightarrow{}\mathbb{F}_q^{n-d_H+1}$ where 
        $\varphi(c_1, c_2, \dots,c_n) = (c_1,c_2,\dots,c_{n-d_H+1})$. 
        Let $c,c' \in C$ and $\varphi(c)=\varphi(c')$.
        Then $d_H(c, c') \leq d_H-1$.
        But $d_H(C)=d_H$ so $d_H(c, c') \geq d_H$. So $c = c'$. 
        $\varphi$ is injective and $\varphi(C) \subseteq \mathbb{F}_q^{n-d_H+1}$. Then $q^{n-d_H+1}\geq|\varphi(C)|=|C|=q^k$.
        Thus, $d_H\leq n-k+1 $.
    \end{block}

\framebreak

    \begin{block}{Generator Matrix}
        Let $k \leq n$ be positive integers, and $C$ be an $[n, k]$ linear code over $\mathbb{F}_q$. Then, the matrix $\mathbf{G} \in \mathbb{F}_q^{k \times n}$ is \textit{generator matrix of} $C$ if 
        \[
        C = 
        \{\, \mathbf{xG} \mid \mathbf{x} \in \mathbb{F}_q^k\,\},
        \]
        that is the rows of $\mathbf{G}$ forms a basis of $C$.
    \end{block}

    \begin{exampleblock}{Example}
    \[
    G = \begin{pmatrix}
        1 & 0 & 0 & 0 & 0\\
        0 & 1 & 0 & 0 & 0 
    \end{pmatrix} \in \mathbb{F}_2^{2 \times 5}
    \] is a generator matrix of $C$.
    \end{exampleblock}

\framebreak

    \begin{block}{Parity-Check Matrix}
        Let $k \leq n$ be positive integers and $C$ an $[n, k]$ linear code over $\mathbb{F}_q$. Then, the matrix $\mathbf{H} \in \mathbb{F}_q^{(n-k) \times n}$ is \textit{parity-check matrix of} $C$ if 
        \[
        C = 
        \{\, \mathbf{y} \in \mathbb{F}_q^n \mid \mathbf{Hy}^T = 0\,\}.
        \]
        $\mathbf{xH}^T$ is called a \textit{syndrome} for any $\mathbf{x} \in \mathbb{F}_q^n$.
    \end{block}

    \begin{exampleblock}{Example}
    \[
    H = \begin{pmatrix}
        0 & 0 & 1 & 0 & 1\\
        0 & 0 & 0 & 1 & 0 \\
        0 & 0 & 0 & 1 & 1
    \end{pmatrix} \in \mathbb{F}_2^{3 \times 5}
    \] is a parity-check matrix of $C$.
    \end{exampleblock}
    
\end{frame}

\subsection{Reed-Solomon Codes}
\begin{frame}

    \begin{block}{Generalized Reed-Solomon code}
        Let $k \leq n \leq q$ be positive integers. Let $\alpha \in \mathbb{F}_q^n$ be an n-tuple of distinct elements and $\beta \in \mathbb{F}_q^n$ be an n-tuple of nonzero elements. The \textit{Generalized Reed-Solomon code} of length $n$ and dimension $k$ is defined as
        \[
        GRS_{n,k}(\alpha, \beta) = \{\, (\beta_1f(\alpha_1), \dots, \beta_nf(\alpha_n)) \mid f \in \mathbb{F}_q[x], deg(f) < k \,\}.
        \]
    \end{block}

    \begin{block}{Reed-Solomon Codes}
        When $\beta = \{\,1, \dots,1 \,\}$, we define \textit{Reed-Solomon codes} as
        \[
        RS_{n,k}(\alpha) = \{\, (f(\alpha_1), \dots, f(\alpha_n)) \mid f \in \mathbb{F}_q[x], deg(f) < k \,\}.
        \]
    \end{block}

    \begin{exampleblock}{Example}
        Let $\alpha = \{\,1,0,2\,\} \in \mathbb{F}_3^3$, $\beta = \{\,2,1,2 \,\} \in \mathbb{F}_3^3$ and $f = x+1 \in \mathbb{F}_3[x]$.\\
        Then
        \[
            GRS_{3,k}(\alpha, \beta) = \{\, (1,1,0)\,\}.
        \]
    \end{exampleblock}
    
\end{frame}

\subsection{Goppa Codes}
\begin{frame}[allowframebreaks]
    Let $m$ be a positive integer and $q$ be an prime number. Let $n = q^m$ and $\mathbb{F}_{q^m}$ be a finite field. Let $G \in \mathbb{F}_{q^m}[x]$. Define the quotient ring as
    \[
    S_m = \mathbb{F}_{q^m}[x] / \langle G \rangle.
    \]
    \begin{block}{Classical Goppa Codes}
    Let $\alpha \in \mathbb{F}_{q^m}^n$ be an n-tuple of distinct elements and $G(\alpha_i) \neq 0$ for all $i \in \{\,1, \dots, n \,\}$. We define the \textit{classical q-ary Goppa codes} as
    \[
    \Gamma(\alpha, \beta) = 
    \{\,c \in \mathbb{F}_q^n \mid \sum_{i = 1}^n \frac{c_i}{x - \alpha_i} = 0 \text{ in } S_m \,\}.
    \]
    \end{block}
    
\framebreak

    \begin{exampleblock}{Example}
        \noindent Choose parameters $[n, m, t]$ as $[3, 3, 2]$.\\
        Represent $\mathbb{F}_8 \cong \mathbb{F}_2[x]/ \langle x^3+x+1 \rangle$. Let $\alpha$ be the root of $x^3+x+1$. So $\mathbb{F}_8 = \{\,0,1,\alpha,\alpha^2,\alpha+1,\alpha^2+\alpha, \alpha^2+\alpha+1,\alpha^2+1\,\}.$\\ 
        Let be the irreducible polynomial $G = x^2+x+1 \in \mathbb{F}_8[x]$.\\
        Construct the quotient ring $S_3 = \mathbb{F}_8/\langle x^2+z+1\rangle.$\\
        Choose a set that avoids the roots of $G$ in $S_3$. Let this set be 
        $G'=\{\,1, \alpha^2,\alpha^2+1\,\}$, then
        \[        
        \noindent G(1)=1,G(\alpha^2)=\alpha+1,
        G(\alpha^2+\alpha+1)=\alpha^2+1,G(\alpha^2+1)=\alpha+1.
        \]\\
    \end{exampleblock}   
    
    \vskip-1em

    \begin{exampleblock}{}
        Parity-check equations over $\mathbb{F}_2$ gives,\\[0.4cm]
        \begin{pmatrix}
            1/G(1) &1/G(\alpha^2)& 1/G(\alpha^2+1)\\
            1/G(1) &\alpha^2/G(\alpha^2)& \alpha^2+1/G(\alpha^2+1)
        \end{pmatrix} = \begin{pmatrix}
            1 & \alpha+1 & \alpha+1\\
            1 & 1 & 1
        \end{pmatrix}.\\[0.4cm]
    \end{exampleblock}
      \end{frame}
    \begin{frame}{}
    \begin{exampleblock}{}
        Expanding to $\mathbb{F}_2$ using 3-bit column vector we get,\\[0.4cm]
        $H$=\begin{pmatrix}
            1&1&1\\
            0&1&1\\
            0&0&0\\
            1&1&1\\
            0&0&0\\
            0&0&0
        \end{pmatrix}.\\[0.4cm]
        Dimension $k = n-rank(H)=3-2=1$. Thus these parameter choice gives a trivial code, $\Gamma(\alpha, \beta) = \{\,(0,0,0)\,\}$. 
    \end{exampleblock}  
\end{frame}

\subsection{Cyclic Codes}
\begin{frame}
    Let $c=\{c_1, c_2, \dots, c_n\} \in \mathbb{F}_q^n$, then its \textit{cyclic shift} is denoted as
    \[
    \sigma(c) = \{c_n, c_1, \dots, c_{n-1}\}.
    \]
    \begin{block}{Cyclic Codes}
        An $[n, k]$ linear code $C$ over $\mathbb{F}_q$ is \textit{cyclic} if $\sigma(C) = C$.    
    \end{block}
    \begin{exampleblock}{Example}
        Let $C$ be a binary code of length 3, where $C = \{\,(0,0,0),(1,1,0),(0,1,1),(1,0,1)\,\}$. All shifts of $C$ are in $C$. Thus, it is a cyclic code.\\
        Our first example is not a cyclic code since,
        \[
        (1,1,0,0,0) \xrightarrow{} (0,1,1,0,0) \notin C.
        \]
        
    \end{exampleblock}
\end{frame}

\subsection{Reed-Muller Codes}
\begin{frame}

    Let $p$ be a prime, $m,r$ positive integers, and $q = p^n$. Let $\mathbb{F}_q[x_1, \dots, x_m]_\leq r$ be the $\mathbb{F}_q$-vector space of polynomials with $m$ variables and degree at most $r$. Fix an order $\{\, \alpha_1, \dots, \alpha_{q^m} \,\}$ of $\mathbb{F}_q^m$.

    \begin{block}{Reed-Muller Codes}
        The \textit{Reed-Muller} code $RM_q(m, r)$ over $\mathbb{F}_q$ is the image of the map
        \[
        ev:\mathbb{F}_q[x_1, \dots, x_m]_\leq r \xrightarrow{} \mathbb{F}_q^{q^m}
        \]
        where
        \[
        ev(f) = (f(\alpha_1), \dots, f(\alpha_{q^m})).
        \]
    \end{block}

    
    
\end{frame}


\subsection{Code Equivalence}
\begin{frame}[allowframebreaks]
    \begin{block}{Isometry}
        Let $V$ be a space with distance $d$ invariant. Linear map $\varphi: V \xrightarrow{}V$ is an \textit{isometry} if for all $\mathbf{x}, \mathbf{y} \in V$ we have $d(\mathbf{x}, \mathbf{y}) = d(\varphi(\mathbf{x}), \varphi(\mathbf{y}))$. 
    \end{block}

    \begin{block}{Code Equivalence}
        Let $C_1, C_2 \subseteq V$ be linear codes. $C_1$ is \textit{equivalent} to $C_2$ if there is an isometry $\varphi \in I_d$ such that $\varphi(C_1) = \varphi(C_2)$.
    \end{block}

    \begin{block}{Permutation Equivalence}
        Code $C_1$ is \textit{permutation equivalent} to $C_2$ if there exists $\sigma \in S_n$ with $\sigma(C_1) = C_2$.   
    \end{block}
    
\framebreak

    \begin{exampleblock}{Example}
        Consider the codes $C_1 \subseteq \mathbb{F}_3^3$ generated by $\mathbf{G}_1 = \begin{pmatrix}
        1 & 0 & 2 \\ 0 & 1 & 1
    \end{pmatrix}$ and $C_2 \subseteq \mathbb{F}_3^3$ generated by $\mathbf{G}_2 = \begin{pmatrix}
        1 & 0 & 1 \\ 0 & 1 & 0
    \end{pmatrix}$. Are these codes permutation equivalent?
    \end{exampleblock}

    \begin{block}{Solution}
    \[C_1 = \{\,(0,0,0),(1,1,0),(1,2,1),(2,1,2),(2,2,0),(0,1,1),(1,0,2),(0,2,2),(2,0,1) \,\}\\
    C_2 = \{\,(0,0,0),(1,1,1),(1,2,1),(2,1,2),(2,2,2),(0,1,0),(0,2,0),(2,0,2),(2,0,1) \,\}.
    \]\\
    $C_1$ and $C_2$ are not permutation equivalent since there is no $\sigma \in S_3$ with $\sigma(C_1)=C_2$.
    \end{block}

\framebreak
    
    \begin{block}{Proposition 2}
        Let $C_1, C_2 \subseteq \mathbb{F}_q^n$ be permutation equivalent. Then for any generator matrix $\mathbf{G_1}$ of $C_1$ and $\mathbf{G_2}$ of $C_2$, there exist a $n \times n$ permutation matrix $\mathbf{P}$ and an invertible matrix $\mathbf{S} \in GL_k(\mathbb{F}_q)$ such that
        \[
        \mathbf{SG_1P} = \mathbf{G_2}.
        \]
    \end{block}

    \begin{block}{Proof}
        Let $C_1$ and $C_2$ be permutation equivalent. Then $\sigma(C_1) = C_2$ for some $\sigma\in S_n$.\\Let $G_1$ be the generator matrix of $C_1$ so that we have $C_2 = \{\mathbf{x}G_1P \mid\mathbf{x}\in \mathbf{F}_n^k\}$ where $P$ is a permutation matrix of $\sigma$. Thus, $G_1P$ is a generator matrix of $G_2$\\ We can obtain a generator matrix of $G_2$ of $C_2$ by using a different basis for $G_1P$. Thus we can find an invertible matrix $S \in GL_k{\mathbb({F}_q)}$ such that $G_2 = S(G_1P)$.
    \end{block}
    
\framebreak

    \begin{exampleblock}{Example}
        We can also check the equivalence of $C_1$ and $C_2$ with this proposition.
        Let $P$ be the permutation matrix representing $\sigma(12)$ so that
        \[
        P = \begin{pmatrix}
            0 & 1 & 0 \\1 & 0 & 0 \\0 & 0 & 1
        \end{pmatrix}.\\
        \text{ Then }G_1P = \begin{pmatrix}
            1&2&0\\0&1&1
        \end{pmatrix}.
        \]\\
        We want such $S \in GL_2(\mathbb{F}_3)$ so that $\begin{pmatrix}
            s_1&s_2\\s_3&s_4
        \end{pmatrix}\begin{pmatrix}
            1&2&0\\0&1&1
        \end{pmatrix} = \begin{pmatrix}
            1&0&1\\0&1&0
        \end{pmatrix}$\\
        So $s_1=s_4=1$, $s_2=s_3=0$, $2s_1+s_2=1$, $2s_3+s_4=0$, and $s_1s_4-s_2s_3 \neq0$.
        No such solution exists. \\We repeat the same processes for all $3!=6$ permutation matrices and obtain no solution.        
    \end{exampleblock}
    
\end{frame} 


\begin{frame}{References}
    \nocite{*}   
    \bibliographystyle{plain}
    \bibliography{ref}      
\end{frame}




\begin{frame}
    \frametitle{Thank You}
    \begin{center}
\Large Thank you for your attention! Questions?
    \end{center}
\end{frame}



\end{document}
