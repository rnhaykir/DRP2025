\begin{titlepage}
\vbox{ }

\vbox{ }

\begin{center}
% Upper part of the page
\textsc{\LARGE DRP Türkiye 2025}\\[1.5cm]

\vbox{ }
% Title
{ \huge \bfseries An Introduction on Algebraic Coding Theory As A Basis For Code-based Cryptography}\\[0.5cm]
{\large June-August, 2025}\\[0.5cm]
% Author and supervisor
\begin{minipage}{0.5\textwidth}
\begin{flushleft} \large
\emph{Mentees:}\\
Ronahi Haykır\\ 
\end{flushleft}
\end{minipage}
\begin{minipage}{0.4\textwidth}
\begin{flushright} \large
\emph{Mentor:} \\
Dr. Yağmur Çakıroğlu\\
\end{flushright}
\end{minipage}
\\[2cm]
% Abstract
\begin{abstract}
Code-based cryptography is one of the main candidates of post-quantum cryptography, that is, to prevent the quantum computers from solving an integer factorization or the discrete logarithm problem over an elliptic curve or over a finite field. It uses hard problems from algebraic coding theory. The problems usually contain decoding a random linear code that is NP-hard. Our aim is to equip the reader with a general knowledge about the algebraic coding theory and the mathematical basis of code-based cryptography mostly focusing on Hamming-metric codes and mentioning rank-metric codes. This section introduces general terminology and important varieties of codes.
\end{abstract}

\vfill
% Bottom of the page
\end{center}
\end{titlepage}